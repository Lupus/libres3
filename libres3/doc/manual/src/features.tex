% !TEX root = manual.tex
\chapter{Highlighted features}
\section{Virtual hosting}
\label{sec:virtualhosting}

Buckets in LibreS3 can be accessed in three ways:
\begin{description}
\item [path-style] \verb|libres3.example.com/bucket/object|

 A direct access to the object, which works even if the bucket name is not DNS compatible.
\item [virtual-hosted subdomain style] \verb|bucket.libres3.example.com/object|

 Requires that bucket names are DNS compatible and a DNS/hosts entry for each
  bucket or a wildcard DNS record (\seeref{sec:dnszone})
\item [virtual-hosted FQDN] \verb|bucket.example.net/object|

  Equivalent to \verb|bucket.example.net.libres3.example.com|.

  The client should send a \verb|Host: bucket.example.net| header, which usually requires a
  \verb|CNAME| DNS entry, \seeref{sec:dnszone}.
  Requires that bucket names are DNS compatible, and that you use your FQDN as
  the bucket name, for example:

 \verb|s3cmd mb s3://bucket.example.net|
\end{description}

The latter can be useful for public buckets, \seeref{sec:pubvirtualhosting}.

\section{Streaming and range-requests}
\label{sec:streaming}

LibreS3 supports streaming files from \SX by downloading blocks as they are
needed (streaming a large file begins as soon as the first batch of blocks is
downloaded by LibreS3).

It is also possible to begin streaming a file from an arbitrary position, and
retrieve less than the entire file, i.e.\ range requests.

This works both for authenticated  (\verb|s3cmd get --continue|) and public
requests (\verb|wget --continue|), \seeref{sec:streaming-videos}.

\section{Access logs}

Since version \verb|1.1| LibreS3 uses the Apache/NCSA Combined Log Format for
its \verb|access.log| file, for example:
\begin{lstlisting}
196.168.1.2 - edwin [06/Jul/2015:08:38:20 +0200] "GET /volpublic/?policy HTTP/1.1" 200 190 "-" "Boto/2.34.0 Python/2.7.9 Linux/3.16.0-4-amd64"
192.168.1.2 - - [06/Jul/2015:09:58:26 +0200] "GET /slider_logo.png HTTP/1.1" 200 11436 "http://volpublic.libres3.example.com:8008/" "Mozilla/5.0 (X11; Linux x86_64; rv:38.0) Gecko/20100101 Firefox/38.0 Iceweasel/38.1.0"
\end{lstlisting}

The first line is an example for an authenticated request, and the second line
is an example for an anonymous request (using directory indexing according to
the referer).

The format is:
\begin{description}
\item[192.168.1.2] IP address of client
\item[-] always unavailable (the \verb|identd| identity)
\item[edwin] name of the authenticated \SX user. If the HTTP result code is
  \verb|401| or \verb|5xx| then it cannot be trusted because the user may not be
  authenticated yet
\item[[06/Jul/2015:09:58:26 +0200]] the server's local date, time and
  timezone
\item[``GET /volpublic/?policy HTTP/1.1''] the request method, path and
  HTTP version
\item[200] the HTTP result code
\item[190] the size of the reply body sent to the client. Note that if the
  connection is closed/reset then this shows the truncated size, not the size
  that would have been sent. This is shown as \verb|-| when the body is empty.
\item[``-''] no HTTP referer available
\item[``Boto/2.34.0 Python/2.7.9 Linux/3.16.0-4-amd64'']

 User-Agent of the client (not all S3 clients send a user-agent, for example \verb|s3cmd| doesn't)
\end{description}

Access.log entries are written asynchronously, after the entire reply body is
sent to the client.

\section{Listing buckets}
\label{sec:listing-buckets}

\verb|s3cmd ls s3://| shows only the buckets that are owned by your user,
what is consistent with S3's behaviour. In case you have access to some buckets
in read or write mode and would like to use them with s3cmd or other clients,
you can enable listing of all accessible buckets by adding the following to
\verb|libres3.conf| and restarting LibreS3:
\begin{lstlisting}
allow_list_all_volumes=true
\end{lstlisting}

\section{IPv6 support}

LibreS3 supports IPv6 communication with \SX version \verb|1.2| or newer
(note that an \SX node can have either an IPv4 or an IPv6 public address, but not both at the same time).
LibreS3 can be reached either by IPv4 or IPv6, depending on how your DNS
wildcard record is set up, and whether your S3 client and OS support IPv6. 

\section{Quota support}

Each volume in \SX has a maximum size (its quota).
There is no equivalent concept in S3, thus you need to specify the default
volume size (and replica count) for buckets created through S3 in \verb|libres3.conf| using the
\verb|volume_size| field. This is done automatically if you used
\verb|libres3_setup| to configure LibreS3.

In case a new file would exceed the \SX volume's quota, LibreS3 should report an
HTTP 413 error. The quota errors will be reported for both normal and multipart
uploads, however in the case of multipart ones, the quota error will only be
reported after all parts have been uploaded to LibreS3.

LibreS3 creates an internal volume for each user to store the parts uploaded
during a multipart upload. This volume's name is prefixed by \verb|libres3-|
and is created with the size specified in the
\verb|volume_size| field and replica count 1. You have to ensure that the size is sufficient to
hold all in-progress (and interrupted) multipart uploads for a user, otherwise
one may run out of space during multipart upload even if the destination volume
has plenty of space available.

Interrupted multipart uploads are not automatically removed (because one can
resume by providing the upload ID), to clean them first run
\verb|s3cmd multipart| on each bucket in turn to find out the objects and upload
IDs, and \verb|s3cmd abortmp| on each upload ID in turn to actually delete it:

\begin{lstlisting}
$ s3cmd multipart -c s3://bucket
$ s3cmd abortmp s3://bucket/object Id
\end{lstlisting}
%%% Local Variables:
%%% mode: latex
%%% TeX-master: "manual"
%%% End:
