% !TEX root = manual.tex
\chapter{Installation}

LibreS3 is regularly tested on Linux and FreeBSD\@. We recommend using the
binary packages from \url{http://www.skylable.com/download} if your platform
is supported.

\section{Docker}

Stable releases of LibreS3 are available as tags on the Docker hub, e.g.:
\begin{lstlisting}
   docker pull skylable/libres3:release-1.2
\end{lstlisting}

Latest master is always available as:
\begin{lstlisting}
   docker pull skylable/libres3:latest
\end{lstlisting}

Follow the instructions on the Docker hub for configuring and running
this container: \url{http://hub.docker.com/r/skylable/libres3}

\section{Binary packages}

\subsection{Debian Wheezy and Jessie}
Add the following entry to \path{/etc/apt/sources.list.d/skylable.list}:
\begin{lstlisting}
deb http://cdn.skylable.com/debian wheezy main
\end{lstlisting}
then run the following commands:
\begin{lstlisting}
# curl 'https://pgp.mit.edu/pks/lookup?op=get&search=0x5377E192B7BC1D2E' | sudo apt-key add -
# apt-get install libres3
\end{lstlisting}

\subsection{CentOS 6/7}
Create the file \path{/etc/yum.repos.d/skylable-sx.repo} with this content:
\begin{lstlisting}
[skylable-sx]
name=Skylable SX
baseurl=http://cdn.skylable.com/centos/$releasever/$basearch
enabled=1
gpgcheck=0
\end{lstlisting}
then execute:
\begin{lstlisting}
# yum install libres3
\end{lstlisting}

\subsection{Fedora 21+}
Create the file \path{/etc/yum.repos.d/skylable-sx.repo} with this content:
\begin{lstlisting}
[skylable-sx]
name=Skylable SX
baseurl=http://cdn.skylable.com/fedora/$releasever/$basearch
enabled=1
gpgcheck=0
\end{lstlisting}
then execute:
\begin{lstlisting}
# yum install libres3
\end{lstlisting}

\section{Source code}

On most Unix platforms you can compile LibreS3
from source. You will need the following packages to be installed together with their
development versions:
\begin{itemize}
    \item OCaml (>= 3.12.1)
    \item camlp4 (matching your OCaml compiler version)
    \item OpenSSL
    \item PCRE C library
    \item GNU Make and m4
    \item Zlib
    \item pkg-config
\end{itemize}
For example, on Debian run:

\begin{lstlisting}
# apt-get install ocaml-native-compilers camlp4-extra libssl-dev libpcre3-dev zlib1g-dev\
 pkg-config make m4
\end{lstlisting}

On Fedora run:

\begin{lstlisting}
# yum install ocaml /usr/bin/camlp4of /usr/bin/camlp4rf /usr/bin/camlp4 openssl-devel\
 pcre-devel zlib-devel pkgconfig make m4 ncurses-devel
\end{lstlisting}


\subsection{Compilation}

Follow the standard installation procedure to install
LibreS3 into the default location (\path{/usr/local}):

\begin{lstlisting}
$ ./configure && make && make check
# make install
\end{lstlisting}
The rest of the manual assumes that LibreS3 was installed from a binary
package, so some paths may be different.

Note: On OpenSolaris/OmniOS you need some additional flags:
\begin{lstlisting}
$ CPPFLAGS=-m64 ./configure --destdir=${DESTDIR} && make && make check
# make install
\end{lstlisting}
%%% Local Variables:
%%% mode: latex
%%% TeX-master: "manual"
%%% End:
